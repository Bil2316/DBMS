\subsection{Kết luận}
Hệ thống đã được hiện thực đầy đủ các yêu cầu chức năng đặt ra, cũng như là các loại truy vấn phức tạp như query with a single condition, query with a composite condition, query with a join, và query with aggregation functions đều đã được áp dụng linh hoạt để xử lý dữ liệu một cách hiệu quả. Tuy nhiên, chỉ riêng query with a subquery không được sử dụng, vì MongoDB không hỗ trợ subquery theo cách truyền thống mà thay vào đó, sử dụng Aggregation Pipeline và các giai đoạn như \$lookup, \$group, và \$match để đạt được kết quả tương đương. Điều này là do MongoDB được thiết kế để xử lý dữ liệu theo hướng document và pipeline thay vì subquery dạng lồng như SQL.

Ngoài ra, giao diện người dùng của hệ thống được thiết kế thông minh, tối giản và dễ sử dụng, tập trung vào việc tối ưu trải nghiệm người dùng. Các thao tác như tìm kiếm, lọc, phân trang, và quản lý dữ liệu được tích hợp mượt mà, giúp hệ thống không chỉ mạnh mẽ về mặt chức năng mà còn thân thiện với người dùng.