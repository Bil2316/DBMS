\indent Sau quá trình nghiên cứu và thực hiện đề tài, nhóm đã hoàn thành các mục tiêu đề ra và đạt được kết quả mong muốn. Thông qua việc tìm hiểu các khía cạnh quan trọng của hệ quản trị cơ sở dữ liệu (DBMS) như lưu trữ và quản lý dữ liệu, chỉ mục hóa, xử lý truy vấn, giao dịch, kiểm soát đồng thời, và sao lưu \& khôi phục dữ liệu, nhóm đã phân tích và so sánh chi tiết giữa hai hệ thống PostgreSQL và MongoDB. Kết quả nghiên cứu cho thấy PostgreSQL nổi bật với khả năng xử lý dữ liệu quan hệ phức tạp nhờ kiến trúc mạnh mẽ và hỗ trợ tốt các tiêu chuẩn SQL, trong khi MongoDB, với kiến trúc NoSQL, thể hiện sự linh hoạt và hiệu quả vượt trội khi làm việc với dữ liệu phi cấu trúc hoặc có khối lượng lớn. Trên cơ sở đánh giá ưu nhược điểm của hai hệ thống trên, nhóm đã lựa chọn MongoDB để phát triển ứng dụng Việc làm, một ứng dụng nhắm đến việc quản lý và tìm kiếm thông tin việc làm với yêu cầu xử lý dữ liệu linh hoạt và khả năng mở rộng cao. Ứng dụng đã được triển khai thành công, tận dụng ưu điểm của MongoDB trong việc lưu trữ và xử lý dữ liệu không có cấu trúc, đồng thời đảm bảo hiệu năng cao khi làm việc với khối lượng dữ liệu lớn.u. Qua đề tài này, nhóm không chỉ củng cố kiến thức về các hệ quản trị cơ sở dữ liệu mà còn nắm bắt được quy trình thiết kế và phát triển một ứng dụng thực tế, từ đó rút ra nhiều kinh nghiệm quý báu trong việc chọn lựa và áp dụng công nghệ phù hợp với yêu cầu cụ thể.