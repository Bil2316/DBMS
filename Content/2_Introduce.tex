Trước khi tiến hành so sánh, chúng ta cần làm rõ khái niệm về SQL và NoSQL.

SQL là một ngôn ngữ truy vấn cấu trúc, được sử dụng để giao tiếp và quản lý cơ sở dữ liệu quan hệ (RDBMS). Cơ sở dữ liệu quan hệ tổ chức dữ liệu trong các bảng, với các mối quan hệ rõ ràng giữa các bảng thông qua khóa chính và khóa ngoại. SQL cung cấp các lệnh như SELECT, INSERT, UPDATE, và DELETE để thực hiện các thao tác trên dữ liệu. Các hệ quản trị cơ sở dữ liệu như MySQL, PostgreSQL, và SQL Server đều sử dụng SQL. SQL là lựa chọn phổ biến khi cần lưu trữ và xử lý các dữ liệu có cấu trúc, có tính nhất quán và quan hệ chặt chẽ.

NoSQL là một thuật ngữ dùng để chỉ các hệ quản trị cơ sở dữ liệu không dựa trên mô hình quan hệ. Các cơ sở dữ liệu NoSQL được thiết kế để xử lý dữ liệu không có cấu trúc, có khả năng mở rộng và có tính linh hoạt cao hơn so với SQL. Các loại cơ sở dữ liệu NoSQL bao gồm cơ sở dữ liệu tài liệu (Document Databases), cơ sở dữ liệu cột (Column-family Databases), cơ sở dữ liệu đồ thị (Graph Databases), và cơ sở dữ liệu kho khóa-giá trị (Key-Value Stores). MongoDB, Cassandra, và Redis là những ví dụ nổi bật của các cơ sở dữ liệu NoSQL. NoSQL thường được sử dụng cho các ứng dụng cần xử lý lượng dữ liệu lớn, có tính chất phi cấu trúc hoặc có yêu cầu mở rộng quy mô nhanh chóng.

\subsection{PostgreSQL}
\indent Trong các hệ quản trị cơ dở dữ liệu, PostgreSQL nổi bật là một hệ quản trị cơ sở dữ liệu quan hệ mã nguồn mở, được phát triển từ năm 1986 và hiện nay trở thành một trong những hệ quản trị cơ sở dữ liệu phổ biến nhất thế giới. PostgreSQL hỗ trợ SQL hoàn chỉnh và có khả năng mở rộng mạnh mẽ. Nó hỗ trợ các tính năng tiên tiến như các kiểu dữ liệu tùy chỉnh, truy vấn đệ quy, và các giao dịch phức tạp. PostgreSQL cũng cung cấp khả năng đồng thời cao, bảo mật tốt và dễ dàng tích hợp với các ứng dụng phát triển. Ngoài ra, nó còn hỗ trợ các tính năng như JSON và XML, giúp PostgreSQL có thể xử lý cả dữ liệu có cấu trúc và phi cấu trúc. PostgreSQL là lựa chọn lý tưởng cho các ứng dụng yêu cầu tính toàn vẹn dữ liệu cao và hiệu suất mạnh mẽ, chẳng hạn như hệ thống tài chính, quản lý chuỗi cung ứng, và các ứng dụng phân tích dữ liệu.

\subsection{MongoDB}
\indent MongoDB là một trong những hệ quản trị cơ sở dữ liệu NoSQL phổ biến nhất, là một cơ sở dữ liệu NoSQL mã nguồn mở, được thiết kế để xử lý các dữ liệu phi cấu trúc và có thể mở rộng quy mô linh hoạt. Thay vì lưu trữ dữ liệu trong các bảng như trong SQL, MongoDB lưu trữ dữ liệu dưới dạng các tài liệu JSON (hoặc BSON), giúp cho việc lưu trữ và truy vấn dữ liệu trở nên linh động và dễ dàng hơn. MongoDB hỗ trợ các tính năng mạnh mẽ như phân vùng dữ liệu (sharding), sao chép (replication), và chỉ mục tìm kiếm phức tạp. Với khả năng mở rộng quy mô tốt và tốc độ truy vấn nhanh, MongoDB rất phù hợp cho các ứng dụng web, mobile, và các hệ thống có dữ liệu phi cấu trúc hoặc thay đổi nhanh chóng, chẳng hạn như các ứng dụng mạng xã hội, dịch vụ trực tuyến, và các hệ thống phân tích dữ liệu lớn.

\subsection{Mục tiêu đề tài}
\begin{itemize}
    \item Nghiên cứu lý thuyết:
        \begin{itemize}
            \item Tìm hiểu và phân tích các chủ đề quan trọng liên quan đến hệ quản trị cơ sở dữ liệu (DBMS), bao gồm:
                \begin{itemize}
                    \item Lưu trữ và quản lý dữ liệu (Data storage \& management).
                    \item Chỉ mục hóa (Indexing).
                    \item Xử lý truy vấn (Query processing).
                    \item Giao dịch (Transaction).
                    \item Kiểm soát đồng thời (Concurrency control).
                    \item Sao lưu và khôi phục dữ liệu (Data backup and recovery).
                \end{itemize}
            \item Thực hiện nghiên cứu chi tiết trên hai hệ quản trị cơ sở dữ liệu phổ biến: PostgreSQL và MongoDB, nhằm so sánh đặc điểm và hiệu quả của chúng.
        \end{itemize}
    \item Phát triển ứng dụng: Xây dựng một ứng dụng Việc làm để minh họa dựa trên một trong hai DBMS phù hợp với yêu cầu miền ứng dụng.
\end{itemize}
